{\color{gray}\hrule}
\begin{center}
\section{Methodology}
\bigskip
\end{center}
{\color{gray}\hrule}
\begin{multicols}{2}
This study employs simulation-based analysis to investigate and compare the behavior of ray and wave optics in two specific imaging configurations: diffraction through a circular aperture and off-axis imaging with a high-aperture spherical lens. 
The simulations were implemented in Python, and all parameters were normalized where appropriate to emphasize physical
 effects without computational bias. The theoretical setups for each experiment follow the configurations detailed in the appendix.

\subsection{Ray Optics Simulation}

In the ray optics approach, light is modeled as straight lines obeying geometric rules. 
The simulations make use of Snell's law and ray propagation through lenses using paraxial approximations. 
Two cases were examined:

\begin{itemize}
\item \textbf{Diffraction-Limited Focus Without Aberrations:} Rays are emitted from a point source and traced through an ideal thin lens. 
The rays converge perfectly at the geometric focus, and no wave-like behavior such as diffraction or interference 
is considered. This represents the idealized case predicted by geometric optics.

\item \textbf{Aberrated Off-Axis Imaging:} Rays from a point source located off the optical axis are propagated
through a high numerical aperture (NA) spherical lens. Due to the curvature of the lens and the off-axis origin, 
the rays fail to converge to a single point, revealing classical spherical aberration. The resulting spot diagram 
illustrates ray spread and loss of resolution.
\end{itemize}

\subsection{Wave Optics Simulation}

The wave optics framework models light as an electromagnetic wave governed by the scalar diffraction theory. 
To account for diffraction and interference, the angular spectrum and Fourier optics approach were used. 
The simulations explored the following:

\begin{itemize}
\item \textbf{Diffraction through a Circular Aperture:} A monochromatic point source is passed through a circular aperture. The resulting point spread function (PSF) is calculated using the Fraunhofer diffraction integral, which in the far-field limit reduces to the Airy pattern:

\[
I(r) = I_0 \8{\frac{2J_1(kr)}{kr}}^2
\]

where $J_1$ is the first-order Bessel function, $k = \frac{2\pi}{\lambda}$, and $r$ is the radial distance in the image plane. 
This model predicts the central bright disk and surrounding diffraction rings.

\item \textbf{Aberrations in High-NA Lens:} A wavefront with a phase distortion, corresponding to spherical aberration, is propagated through a lens aperture. The aperture function includes a phase delay term of the form:

\[
P(r) = A(r) \exp\left(i k W(r)\right)     
\]

where $W(r)$ represents the aberration function and $A(r)$ the amplitude profile across the aperture. The resulting PSF is computed using a two-dimensional Fourier transform, which reveals distortions and asymmetry in the focal intensity distribution due to interference and phase errors.
\end{itemize}

All simulations were executed using discretized spatial grids and FFT-based propagation techniques, which enabled efficient modeling of diffraction and aberration phenomena. The results were visualized to directly compare the predictions from ray and wave optics under matched conditions.
\end{multicols}
