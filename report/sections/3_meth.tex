{\color{gray}\hrule}
\begin{center}
\section{Methodology}
\bigskip
\end{center}
{\color{gray}\hrule}
\begin{multicols}{2}
This papers uses simulation-based analysis to compare the behaviour of ray and wave optics in two specific imaging configurations: diffraction through a circular aperture and off-axis imaging with a high-aperture spherical lens. 
The simulations were implemented in Python; the theoretical setups for each experiment follow the configurations detailed in the abstract.

\subsection{Ray Optics Simulations}
In the ray optics approach, light is modeled as straight lines obeying geometrical rules. 
The simulations make use of Snell's law and ray propagation through lenses using paraxial approximations. 
Two cases were examined:

\begin{itemize}
\item \textbf{Diffraction-less Focus Without Aberrations:} Rays are emitted from a point source and traced through an ideal thin lens. 
The rays converge perfectly at the focus, and no wave-like phenomena such as diffraction or interference 
are considered. This represents the idealised case predicted by geometric optics.

\item \textbf{Aberrated Off-Axis Imaging:} Rays from a point source located off the optical axis are propagated
through a high numerical aperture (NA) spherical lens. Due to the curvature of the lens and the off-axis origin, 
the rays fail to converge to a single point, revealing classical spherical aberration. The resulting spot diagram 
showcases ray spread and loss of resolution.
\end{itemize}

\subsection{Wave Optics Simulation}

The wave optics framework models light as an electromagnetic wave governed by diffraction theory. 
To account for diffraction and interference, Fourier optics was used. 

\medskip
The simulations explored the following:

\begin{itemize}
\item \textbf{Diffraction through a Circular Aperture:} A point source is passed through a circular aperture. The resulting point spread function (PSF) is calculated using the approximate Equation (1) discussed earlier.


\item \textbf{Aberrations in High-NA Lens:} A wavefront with a phase distortion, corresponding to spherical aberration, is propagated through a lens aperture. The aperture function includes a phase delay term of the form:

\[
P(r) = A(r) \exp\left(i k W(r)\right)     
\]

where $W(r)$ represents the aberration and $A(r)$ the amplitude across the aperture. The resulting PSF is computed using a 
2D Fourier transform, which shows distortions and asymmetry in the focal intensity distribution due to interference and phase errors.
\end{itemize}

All simulations were executed using discretised grids and FFT-based propagation techniques, as explained above. 
The results were visualised to directly compare the predictions from ray and wave optics under similar/matched conditions.
\end{multicols}
