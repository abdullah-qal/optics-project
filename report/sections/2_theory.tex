{\color{gray}\hrule}
\begin{center}
\section{Theory and Literature Review}
\bigskip
\end{center}
{\color{gray}\hrule}
\begin{multicols}{2}
\subsection{Fundamental Equations and Models}
Ray optics simplifies light behavior by modeling it as straight-line rays and neglecting diffraction and interference,
as discussed previously. The model is governed by Fermat's Principle, which states that light follows the path of least time.
From this, fundamental geometric rules like Snell's Law for refraction arise: 
\[
n_1\6\sin{\theta_1} = n_2\6\sin{\theta_2}
\]
Under small-angle (paraxial) conditions, ray propagation through optical components can be described using ray transfer matrices (ABCD matrices): \[
\begin{bmatrix}
x_2 \\ \theta_2
\end{bmatrix} = \begin{bmatrix} A & B \\ C & D \end{bmatrix} \begin{bmatrix} x_1 \\ \theta_1 \end{bmatrix} 
\]
On the other hand, wave optics treats light as a wave, enabling accurate modeling of interference and diffraction. Wavefront propagation 
is described by the Huygens-Fresnel principle, and image formation is often analysed using Fourier optics. 
The light field at an observation point $P$ can be calculated using the diffraction integral:
\[
U(P) = \4{e^{jkz}}{j\lambda z} \iint\limits_{\R^2} U(Q) e^{\5{jk\9{(x-x')^2 + (y-y')^2}}{2z}} \dd A'
\]
Key metrics in wave optics include the PSF (Point Spread Function), which explains how a point source is imaged, and the MTF (Modulation Transfer Function),
 which showcases the system response to different spatial frequencies.
\end{multicols}
