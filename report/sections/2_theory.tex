{\color{gray}\hrule}
\begin{center}
\section{Theory and Literature Review}
\bigskip
\end{center}
{\color{gray}\hrule}
\begin{multicols}{2}
\subsection{Fundamental Equations and Models}
Ray optics simplifies light behavior by modeling it as straight-line rays and neglecting diffraction and interference,
as discussed previously. The model is governed by Fermat's Principle, which states that light follows the path of least time.
From this, fundamental geometric rules like Snell's Law for refraction arise: 
\[
n_1\6\sin{\theta_1} = n_2\6\sin{\theta_2}
\]
Under small-angle (paraxial) conditions, ray propagation through optical components can be described using ray transfer matrices (ABCD matrices): \[
\begin{bmatrix}
x_2 \\ \theta_2
\end{bmatrix} = \begin{bmatrix} A & B \\ C & D \end{bmatrix} \begin{bmatrix} x_1 \\ \theta_1 \end{bmatrix} 
\]
On the other hand, wave optics treats light as a wave, enabling accurate modeling of phenomena such as interference and diffraction,
which are especially important when feature sizes are comparable to the wavelength of light. 
One key application of wave optics is in understanding the point spread function (PSF), 
how a point source of light is imaged through an optical system.

For a circular aperture, the intensity distribution in the image plane is described by the Airy pattern, given by:
\[ 
I(r) = I_0\8{\4{2J_1(kr)}{kr}}^2
\]
This function arises from the Fourier transform of a circular aperture, and it accurately predicts the Airy disk and its 
surrounding diffraction rings, which define the resolution limit of the optical system. This contrasts with ray optics,
which would predict a perfect point image in the absence of aberrations.


\end{multicols}
