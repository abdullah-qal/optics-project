\begin{multicols}{2}
    \tableofcontents
    \section{Introduction}
    \subsection{Background Information}
    Various industries, such as consumer electronics and medicine, rely heavily on optical imaging systems. 
    The main focus of these systems is to explain how light interacts during the process of forming an image, 
    so engineers and scientists would need to use the ideas involving both ray and wave optics to get a clear 
    understanding of the systems at hand. Understanding the distinctions between these models is crucial when 
    optimising imaging performance, especially as technology pushes toward higher precision and smaller scales.
    \subsubsection{Introduction to Ray Optics}
    In ray optics, sometimes called geometric optics, the main principle is that light takes a straight path and 
    obeys the rules of both reflection and refraction. It provides a powerful and intuitive approach for analyzing 
    systems involving lenses, mirrors, and prisms, and is particularly useful for macroscopic optical designs where 
    diffraction effects are negligible. 
    \subsubsection{Introduction to Wave Optics}
    In wave optics, light is modeled as a sinusoidal wave. This is essential for accurately describing phenomena like 
    interference, diffraction, and coherence, which become significant when the wavelength of light is comparable to 
    the size of system features. While the mathematical tools required, such as the Fourier transform, are more complex,
    they enable deeper insights into image formation and point spread functions. 
    
    \subsection{Problem Statement}
    The ray optics model is widely used in designing optical systems due to its simplicity. However, it fails to capture wave-
    based phenomena like diffraction and interference, which are crucial in determining an image quality in real-world systems. 
    This leads to inaccurate predictions in scenarios involving fine resolution or high numerical apertures. Due to that, there is 
    thus a need to explore the extent and nature of those limitations through comparative analysis with wave optics. 
    \subsection{Objectives}
    The main objectives of this study is to: 
        \begin{itemize}
            \item Examine how ray and wave optics differ in predicting image formation. 
            \item Investigate specific phenomena, diffraction and aberrations, that challenge the ray model. 
            \item Assess the implications of these differences on image quality and optical design.
        \end{itemize}
    \subsection{Scope}
    This study is limited to two common but distinguishing optical scenarios: diffraction at a circular aperture, and off-axis imaging in high-NA lenses. 
    Simulations are conducted using idealised models to isolate the effects of interest. Other factors such as polarisation, nonlinear effects, or material dispersion are beyond
    the scope of this work.
\end{multicols}
