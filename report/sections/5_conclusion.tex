{\color{gray}\hrule}
\begin{center}
\section{Conclusions}
\bigskip
\end{center}
{\color{gray}\hrule}
\begin{multicols}{2}
The study explored two optical models, ray optics and wave optics, in the context of imaging systems, through simulation setups
of diffraction and aberration scenarios. Ray optics provide a simplified and intuitive estimate of image formation, operating
under the paraxial approximation, which correctly illustrates focal convergence and spherical aberration effects under ideal
assumptions. On the other hand, wave optics showed a more accurate picture by including diffraction and interference, which ray
optics neglects. The simulation of a point source provided an Airy disk, which reflected the resolution limits due to aperture
diffraction. In the aberration case, a wave model captured asymmetric distortions in the PSF that ray models could not fully
describe. Overall, ray optics is useful for fast approximation and conceptual insight; wave optics is essential for accurate
analysis in systems where diffraction and phase effects are non-negligible. Their combined use offers a deeper understanding of
image formation and the physical limits of optical systems.

\end{multicols}

All work files are avaible in the GitHub repository: \url{https://github.com/abdullah-qal/optics-project}