This paper aims to provide a comparative analysis of ray (geometric) optics with wave (physical) optics in the context of 
imaging systems. We evaluate how well each explains or predicts how images are formed and what limits the quality of those images in physical optical systems.   
More specifically, we highlight: 
\begin{itemize}
    \item Limitations of the ray model in representing physical phenomena accurately.
    \item How the wave model explains phenomena that affect image sharpness, brightness, and contrast.
\end{itemize}
To that end, we consider the following two points of interest under different optical setups by means of a simulation-based approach, and verify
whether the respective ray and wave models' predictions hold:  
\begin{enumerate} 
    \item Diffraction Effects in a Lens-Based Imaging System.
        \begin{itemize}
            \item \tb{Setup:} Simulate a point source imaged through a circular lens aperture. 
            \item \tb{Ray Optics Prediction:} The image of a point is a perfect point at the focus.
            \item \tb{Wave Optics Prediction:} The image is an Airy disk due to diffraction at the aperture edges.
        \end{itemize}
    \item Aberrations in High-Aperture Lenses.
        \begin{itemize}
            \item \tb{Setup:} Simulate off-axis point imaging using a high numerical aperture lens, for simplicity, a spherical lens is considered. 
            \item \tb{Ray Optics Prediction:} Rays don't all focus to a single point due to primarily spherical and chromatic aberration, but can still be traced.
            \item \tb{Wave Optics Prediction:} Also accounts for the interference patterns and phase shifts that occur; blurring and PSF distortions should be more precise. 
        \end{itemize}
\end{enumerate}
The above two setups support our objectives by examining two important optical processes: diffraction and aberrations, both of which critically impact optical system performance. Our simulations confirm 
our predictions for both models and reveal the limitations of the ray model in predicting systems with strong diffractive or aberrative effects.   

